\thispagestyle{empty}

\begin{center}
  {\bf \Huge Ringraziamenti}
\end{center}

\vspace{2cm}

\emph{
Desidero esprimere la mia più sincera gratitudine a tutte le persone che mi hanno supportato e guidato durante il percorso universitario e, in particolare, durante il tirocinio curriculare e la stesura di questa tesi. Un ringraziamento speciale va al mio supervisore, Prof. Alberto Montresor, per la preziosa guida, i consigli e il supporto costante, e al Dott. Cristian Consonni per la disponibilità, la pazienza e l’aiuto offerto lungo tutto il tirocinio. Un pensiero riconoscente va alla mia famiglia, che mi ha permesso di vivere questa esperienza unica, sostenendomi sempre in ogni modo possibile. Desidero ringraziare in modo particolare mia sorella Maddy, che non ha mai fatto mancare la sua presenza nei momenti di difficoltà e di bisogno, e a cui auguro di vivere un percorso universitario altrettanto felice, capace di regalarle gioia e soddisfazioni. Non riuscirò mai a citare tutti gli amici che mi hanno accompagnato in questi anni, ma voglio ricordare quanto sia stato fondamentale il loro sostegno. La vita non è fatta soltanto di studio e lavoro, ma anche (e soprattutto) di momenti di socialità, senza i quali probabilmente sarei impazzito. Non potendoli nominare uno per uno, desidero allora ricordare alcuni luoghi che custodiscono i segni più belli della nostra amicizia: davanti a un impianto di pannelli fotovoltaici, nello studentato, in BUC, al Caimano, in Scaletta, al Melo Mangio, a Povo, a casa di Sarzo, a casa di Lyza, al Monte Calvario, sulle Dolomiti, al Refuel, in piscina da Luca e in giro per l’Europa. Più dei luoghi, però, ciò che davvero conta sono le persone con cui ho vissuto questi momenti, a cui va la mia più sincera gratitudine. Infine, desidero ringraziare i Camosci che, prenotando con anticipo un Airbnb a Trento, mi hanno dato la motivazione necessaria per concludere questa tesi. Un pensiero speciale va anche ai miei coinquilini di Via Tommaso Gar 4 (sia a chi è ancora presente sia a chi è già andato via) con i quali ho condiviso una parte importante della mia vita trentina, sempre ricca di conversazioni stimolanti. È grazie a loro se in casa si respira un’aria serena e allegra. Mi scuso in anticipo se questa sera (16 settembre) farò dei danni alla casa (ovviamente scherzo).
}

